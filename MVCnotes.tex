\documentclass{article}
\usepackage[utf8]{inputenc}
\usepackage{physics}

\title{MVC Notes}
\author{Alex Feng}
\date{February 2023}

\begin{document}
\setlength{\parindent}{0pt}    
\maketitle

\section{Vectors}
$$u \cross v = 
\begin{vmatrix}
    i & j & k \\
    u_1 & u_2 & u_3 \\
    v_1 & v_2 & v_3
\end{vmatrix}
 = i\begin{vmatrix} 
    u_2 & u_3 \\ v_2 & v_3
    
\end{vmatrix}
- j\begin{vmatrix}
    u_1 & u_3 \\
    v_1 & v_3
\end{vmatrix}
+k\begin{vmatrix}
    u_1 & u_2 \\
    v_1 & v_2
\end{vmatrix}$$

$$= i(u_2 v_3 - u_3 v_2) - j(u_1 v_3 - u_3 v_1) + k(u_1 v_2 - u_2 v_1)$$

$$\textnormal{vector projection:  } proj_v u = \frac{u \cdot v}{|v|} \hat{v}$$ 

$$\textnormal{scalar projection:  } |proj_v u| = \frac{u \cdot v}{|v|}$$

\section{3D graphs}

The distance d between a point Q and a line $r=r_0 +tv$
$$d = \frac{|v \cross \Vec{PQ}|}{|v|} $$
where P is any point on the line and v is a vector parallel to the line. This makes sense since $d = |\Vec{PQ}| \sin \theta$

\vskip 30pt

\noindent For a plane $a(x-x_0 ) + b (y- y_0 ) + c(z- z_0) = 0$
the normal vector $n = (a,b,c)$
\pagebreak
\subsection{Quadric Surfaces}
$$\textnormal{Ellipsoid: }\frac{x^2}{a^2} + \frac{y^2}{b^2} + \frac{z^2}{c^2} = 1$$
$$\textnormal{Elliptic paraboloid: }z = \frac{x^2}{a^2} + \frac{y^2}{b^2}$$
$$\textnormal{Hyperboloid of one sheet: }\frac{x^2}{a^2}+\frac{y^2}{b^2}-\frac{z^2}{c^2}=1$$
$$\textnormal{Hyperboloid of two sheets: }-\frac{x^2}{a^2}-\frac{y^2}{b^2}+\frac{z^2}{c^2}=1$$
$$\textnormal{Elliptic cone }\frac{x^2}{a^2}+\frac{y^2}{b^2}=\frac{z^2}{c^2}$$
$$\textnormal{Hyperbolic paraboloid: }z=\frac{x^2}{a^2}-\frac{y^2}{b^2}$$

\section{Vector Valued Functions}
$$\textnormal{unit tangent vector: } \vec{T} = \frac{\vec{v}}{|\vec{v}|}$$
$$\textnormal{principle unit vector: } \vec{N} = \frac{d\vec{T}/dt}{|d\vec{T}/dt|}$$
The principle unit vector is always perpendicular to the unit tangent vector, which makes sense since it points in the rate of change of the unit tangent vector, which can only change in direction, not length
$$\textnormal{curavture: } \kappa = \bigg |\frac{d\vec{T}}{ds} \bigg | = \frac{1}{|\vec{v}|} \bigg | \frac{d\vec{T}}{dt} \bigg | = \frac{|\vec{v} \times \vec{a} |}{ |\vec{v}|^3}$$

$$\textnormal{components of acceleration: } \vec{a} = a_N \vec{N} + a_T \vec{T}$$
$$a_n = \kappa |\vec{v}|^2 = \frac{|\vec{v} \times \vec{a}|}{|\vec{v}|}$$
$$a_T = \frac{d^2 s}{dt^2} = \frac{\vec{v} \cdot \vec{a}}{|\vec{v}|}$$

$$\textnormal{unit binormal vector: } \vec{B} = \vec{T} \times \vec{N} = \frac{\vec{v} \times \vec{a}}{|\vec{v} \times \vec{a}|}$$

$$\textnormal{torsion: } \tau = - \frac{d\vec{B}}{ds} \cdot \vec{N} = \frac{(\vec{v} \times \vec{a}) \cdot \vec{a}'}{|\vec{v} \times \vec{a}|^2} = \frac{(\vec{r}' \times \vec{r}'') \cdot \vec{r}'''}{|\vec{r}' \times \vec{r}''|^2}$$

\section{Derivatives}
Chain Rule for one independent variable:
$$\frac{dz}{dt} = \frac{\partial z}{\partial x} \frac{dx}{dt} + \frac{\partial z}{\partial y} \frac{dy}{dt}$$
Chain Rule for two independent variables:
$$\frac{\partial z}{\partial s} = \frac{\partial z}{\partial x} \frac{\partial x}{\partial s} + \frac{\partial z}{\partial y}\frac{\partial y}{\partial s}$$
$$\frac{\partial z}{\partial t} = \frac{\partial z}{\partial x} \frac{\partial x}{\partial t} + \frac{\partial z}{\partial y}\frac{\partial y}{\partial t}$$
Implicit Differentiation: 
$$\frac{dy}{dx} = -\frac{F_x}{F_y}$$
Gradient in 3 dimensions:
\\For a function $f(x,y,z)$,
$$\nabla f(x,y,z) = f_x \hat{i} + f_y \hat{j} + f_z \hat{k}$$
$\nabla f(x,y,z)$ will be perpendicular to the level surface (3D surface where the value of f(x,y,z) is constant).
\\
\\Second Derivative Test:
\\For a function f(x,y), the discriminant is
$$D(x,y) = \begin{vmatrix} 
    f_{xx} & f_{xy} \\ f_{yx} & f_{yx}
    \end{vmatrix} = f_{xx} f_{yy} - (f_{xy} )^2$$
When $D(a,b) > 0$ then the surface has the same general behavior in all directions near $(a,b)$ (use normal partial derivative to determine if it is min or max). There is a saddle point when $D(a,b) < 0$. The test is inconclusive when $D(a,b) = 0$ 
\\
\\
Lagrange Multipliers:
\\For an objective function $f(x,y,z)$ and constraint $g(x,y,z) = 0$, at a relative extrema $(a,b,c)$ there exists a Lagrange Multiplier, $\lambda$, where
$$\nabla f(a,b,c) = \lambda \nabla g(a,b,c)$$
Intuitively, the  relative extrema is located at a point where objective function surface and the constraint surface are tangent to each other. The gradient for both are orthogonal to the level surface (since it points to the next increasing level surface direction), so the two gradients must be parallel to each other and can be multipled by a scalar to equal each other.   


\end{document}
